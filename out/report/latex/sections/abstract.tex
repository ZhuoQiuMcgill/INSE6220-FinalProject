The Kepler mission has produced a large catalog of Kepler Objects of Interest (KOIs) that contain both genuine transiting exoplanets and a substantial number of false positive signals. To support statistical studies and target selection, the Kepler Robovetter assigns each KOI a disposition score (koi\_score), a continuous value between 0 and 1 that summarizes the confidence that a given signal is a true planet candidate rather than a false positive. Computing this score, however, requires access to the full vetting pipeline and injection--recovery experiments, which are not easily reproduced by end users. In this project, we investigate whether koi\_score can be approximated using only catalog-level physical and observational parameters available in the NASA Exoplanet Archive KOI cumulative table. First, the selected KOI features describing transit shape, signal-to-noise ratio, and stellar properties are standardized and analyzed using principal component analysis (PCA) to reduce dimensionality and reveal dominant directions of variation. The resulting principal components are then used as inputs to regression models, including a baseline linear regression model and a multilayer perceptron (MLP) regressor, to predict koi\_score. Model performance is evaluated on a held-out validation set using standard regression metrics and comparisons between original and PCA-transformed feature spaces. The overall goal is to assess whether a lightweight, data-driven surrogate can capture the main behavior of the complex vetting score, providing a fast proxy for KOI reliability based solely on easily accessible catalog attributes.
