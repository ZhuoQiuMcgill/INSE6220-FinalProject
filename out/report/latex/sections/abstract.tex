We study whether the Kepler Objects of Interest (KOI) vetting confidence score (koi\_score) can be approximated by a lightweight surrogate model. Using the NASA Exoplanet Archive KOI cumulative table, we preprocess physically meaningful features, apply Principal Component Analysis (PCA) to reduce dimensionality and de-correlate inputs, and train regression models (linear regression and a multilayer perceptron). We evaluate models with R$^2$, MSE, RMSE, and MAE on a held-out test set and analyze the effect of using original features versus PCA components. The proposed pipeline aims to provide a quick, reproducible proxy for koi\_score to support screening and prioritization of candidate exoplanets. We outline data cleaning choices, model configurations, and limitations, and discuss how such a surrogate could aid quality-oriented decision-making in KOI triage.
