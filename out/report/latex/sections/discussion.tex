The results suggest how model choice and feature representation affect koi\_score approximation. Linear regression offers transparency and fast training; when PCA reduces multicollinearity, it may stabilize coefficients and improve generalization. The MLP can capture non-linear relationships and may yield lower error if tuned appropriately, at the cost of reduced interpretability.

From a quality-control perspective, a surrogate that is accurate in the high koi\_score range can assist rapid prioritization of likely high-quality KOIs. Nonetheless, limitations remain: koi\_score itself reflects uncertainties and pipeline decisions; models depend on the training distribution; and important signals may reside in attributes not captured by tabular features. Careful monitoring and periodic retraining are recommended if the surrogate is used operationally.
