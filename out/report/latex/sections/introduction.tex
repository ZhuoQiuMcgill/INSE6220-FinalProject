The Kepler mission was designed to measure the occurrence rate of Earth-sized planets in and near the habitable zones of Sun-like stars by continuously monitoring the brightness of over 150{,}000 targets for periodic transit signals.\cite{borucki2010kepler} Over its four-year prime mission, Kepler produced a rich catalog of transit-like events that has transformed our understanding of exoplanet populations. However, not every threshold-crossing event (TCE) corresponds to a genuine transiting planet: many apparent signals arise from eclipsing binaries, background blended stars, or instrumental and data-processing artifacts, and therefore must be vetted and classified as false positives.\cite{coughlin2016dr24}

To systematize vetting at scale, the Kepler project developed the Robovetter, an automated pipeline that applies a battery of metrics and heuristic tests to each TCE and assigns a disposition as planet candidate or false positive.\cite{thompson2018dr25} The resulting Kepler Objects of Interest (KOI) catalogs, together with their dispositions and derived stellar and planetary parameters, are curated and distributed through the NASA Exoplanet Archive.\cite{nasa_koi_docs} In the final Kepler data release (DR25), an additional continuous parameter, the disposition score ($\mathrm{koi\_score}$), was introduced for each KOI.\cite{nasa_koi_score_doc} This score takes values between 0 and 1 and quantifies the Robovetter's confidence in its current disposition: scores near 1 indicate strong confidence that the signal is a true planet candidate, whereas scores near 0 indicate strong confidence in a false-positive classification. The score is computed via a Monte Carlo procedure that repeatedly perturbs the underlying vetting metrics and re-runs the Robovetter many thousands of times, recording the fraction of trials in which the KOI is classified as a candidate.\cite{nasa_koi_score_doc}

Because $\mathrm{koi\_score}$ condenses the complex, multi-step vetting logic into a single continuous reliability metric, it has become an important ingredient in occurrence-rate and population studies based on Kepler data; several works adopt thresholds (e.g., $\mathrm{koi\_score}\ge 0.9$) when constructing high-purity samples of planet candidates.\cite{thompson2018dr25} However, computing $\mathrm{koi\_score}$ requires the full Robovetter software, detailed data-validation products, and large suites of injection--recovery experiments, which are computationally expensive and not easily reproduced by end users who only have access to the catalog-level KOI table. This motivates the question of whether a simpler, purely data-driven surrogate could approximate the disposition score using only the physical and observational parameters already provided in the NASA Exoplanet Archive.

In this project, we address the following problem: given the multivariate catalog of KOI attributes---such as orbital period, transit depth and duration, impact parameter, signal-to-noise ratio, and stellar effective temperature, surface gravity, and radius---can we use statistical methods to reconstruct or approximate the Robovetter disposition score $\mathrm{koi\_score}$? To handle the high dimensionality and strong correlations among these features, we first apply principal component analysis (PCA) to obtain an orthogonal set of components that capture most of the variance in the KOI feature space.\cite{jolliffe2002pca} These principal components are then used as inputs to regression models that map KOI properties to disposition scores. As baselines, we consider classical linear regression models, which offer interpretability in terms of linear combinations of principal components, and compare them to a non-linear multilayer perceptron (MLP) regressor that can capture more complex relationships between features and $\mathrm{koi\_score}$.\cite{goodfellow2016deep} By evaluating these models on held-out data, we quantify how well catalog-level parameters alone can reproduce the behavior of the Kepler vetting pipeline and assess the potential of such surrogate models as lightweight tools for KOI quality and reliability assessment.
