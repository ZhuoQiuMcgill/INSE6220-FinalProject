The Kepler mission has produced a large catalog of Kepler Objects of Interest (KOIs), which include both viable exoplanet candidates and astrophysical or instrumental false positives. Robust vetting is essential to separate likely planets from spurious detections. In this context, the koi\_score is a confidence-like quantity derived by the Kepler vetting processes and related tools (e.g., the Robovetter), indicating the reliability of a KOI.

From a quality-engineering perspective, koi\_score can be viewed as a quality or reliability indicator, while KOI attributes (e.g., transit depth, duration, period, signal-to-noise, stellar parameters) serve as process/product features. Computing the full vetting score may be expensive and not always fully transparent; therefore, we investigate whether a compact surrogate model can approximate koi\_score directly from publicly available KOI features.

This paper makes three contributions: (i) we assemble and preprocess a KOI feature set suitable for regression; (ii) we examine Principal Component Analysis (PCA) to improve interpretability and mitigate multicollinearity; and (iii) we compare linear and non-linear regressors---specifically linear regression and a multilayer perceptron (MLP)---evaluated with standard regression metrics. We also reflect on how such a surrogate can support rapid screening and prioritization of candidate exoplanets in practice.
