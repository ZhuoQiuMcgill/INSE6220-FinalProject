Kepler monitored the brightness of more than 150{,}000 stars to search for periodic dimmings caused by transiting planets.\cite{borucki2010kepler} Over its four-year prime mission, it produced a rich set of transit-like events and reshaped our view of exoplanet populations. Not every threshold-crossing event (TCE) is a real planet, however. Many are eclipsing binaries, blends, or instrumental and processing artifacts, and must be vetted as false positives.\cite{coughlin2016dr24}

To scale this vetting, the project built the Robovetter---an automated pipeline that applies many metrics and heuristic tests to each TCE and assigns a disposition of planet candidate or false positive.\cite{thompson2018dr25} The resulting Kepler Objects of Interest (KOI) catalogs and parameters are distributed by the NASA Exoplanet Archive.\cite{nasa_koi_docs} In the final Kepler release (DR25), the catalog added a continuous disposition score ($\mathrm{koi\_score}$) for each KOI.\cite{nasa_koi_score_doc} The score ranges from 0 to 1 and reflects the Robovetter's confidence: values near 1 indicate a likely planet candidate; values near 0 indicate a likely false positive. It is computed by a Monte Carlo procedure that perturbs the underlying metrics, re-runs the Robovetter thousands of times, and records the fraction of trials labeled as candidate.

Because $\mathrm{koi\_score}$ compresses complex, multi-step logic into one reliability metric, it is widely used in occurrence-rate and population studies. Thresholds such as $\mathrm{koi\_score}\ge 0.9$ are common when building high-purity samples.\cite{thompson2018dr25} Yet reproducing $\mathrm{koi\_score}$ requires the full pipeline, detailed validation products, and large injection--recovery suites---resources most users do not have. This raises a practical question: can a lightweight, data-driven surrogate approximate the disposition score using only the catalog attributes provided in the NASA Exoplanet Archive?

We investigate this question using tabular KOI features such as orbital period, transit depth and duration, impact parameter, signal-to-noise ratio, and basic stellar properties. To handle dimensionality and correlations, we first apply principal component analysis (PCA) to obtain orthogonal components that capture most of the variance.\cite{jolliffe2002pca} We then train regression models to predict $\mathrm{koi\_score}$ from these representations: a linear regression baseline for interpretability and a non-linear multilayer perceptron (MLP) for flexibility.\cite{goodfellow2016deep} Evaluated on held-out data, these models test how well catalog-only inputs can reproduce the behavior of the Kepler vetting pipeline and whether such surrogates can provide fast, practical support for KOI reliability assessment.
